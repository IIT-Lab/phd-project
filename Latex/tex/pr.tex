\documentclass[conference]{IEEEtran}
\IEEEoverridecommandlockouts
% The preceding line is only needed to identify funding in the first footnote. If that is unneeded, please comment it out.
\usepackage{cite}
\usepackage{amsmath,amssymb,amsfonts}
\usepackage{algorithmic}
\usepackage{graphicx}
\usepackage{textcomp}
\usepackage{xcolor}
\def\BibTeX{{\rm B\kern-.05em{\sc i\kern-.025em b}\kern-.08em
    T\kern-.1667em\lower.7ex\hbox{E}\kern-.125emX}}
\begin{document}

\title{Joint Power Allocation and Network Slicing In an End to End O-RAN System
}

\author{\IEEEauthorblockN{1\textsuperscript{st} Mojdeh Karbalaee Motalleb}
\IEEEauthorblockA{\textit{Electrical and Computer Engineering} \\
\textit{Tehran University}\\
Tehran, Iran \\
mojdeh.karbalaee@ut.ac.ir}
\and
\IEEEauthorblockN{2\textsuperscript{nd} Given Name Surname}
\IEEEauthorblockA{\textit{dept. name of organization (of Aff.)} \\
\textit{name of organization (of Aff.)}\\
City, Country \\
email address}

}

\maketitle

\begin{abstract}

\end{abstract}

\begin{IEEEkeywords}
component, formatting, style, styling, insert
\end{IEEEkeywords}

\section{Introduction}
This document is a model and instructions for \LaTeX.
Please observe the conference page limits. 

\section{System Model and Problem Formulation}
In this section, first, we present the downlink (DL) of O-RAN System. Then we obtain achievable rate and delays.
Afterward, the main problem is expressed.
\subsection{System Model}
Suppose that there are $S$ Slices Serving $V$ Services. Each Service $v\in \{1,2,...,V \} $, consists of $U_v$ single antenna users that require certain service. Each slice $s \in \{1,2,...,S \}$ consists of $R_s$ RRHs and $N_s$ PRBs. All the RRHs in a slice that is mapped to a service, transmit signals to all the UEs in specific service. Each RRH $r \in \{1,2,...,R \}$ is connected to BBU pool via an optical fiber link with limited fronthaul capacity. Also each RRH and PRB can serve more than one slice. It is considered that in BBU, the system has 2 processing layer consists of $M_1$ homogeneous VMs in first layer and $M_2$ homogeneous VMs in second layer.
\subsection{Achievable Rate}
In this subsection, the Achievable Rate is obtained as below.
The achievable data rate for $i^{th}$ UE in $v^{th}$ service can be written as 
\begin{equation}\label{eq1}
\mathcal{R}_{u(v,i)} = B \log_2({1+ \rho_{u(v,i)}})
\end{equation}
where $B$ is the bandwidth of system and $\rho_{u(v,i)}$ is the SNR of $i^{th}$ UE in $v^{th}$ service which is obtained from 
\begin{equation}\label{eq2}
\rho_{u(v,i)} =  \frac{P_{u(v,i)}\sum_{s=1}^{N_s}|\bold{h}_{R_s,u(v,i)}^H \bold{w}_{R_s,u(v,i)}|^2 a_{vs}}{BN_0 + I_{u(v,i)}}
\end{equation}
Where, $P_{u(v,i)}$ represents the transmitted power allocated by RRHs to $i^{th}$ UE in $v^{th}$ service. Also, 
$\bold{h}_{R_s,u(v,i)} \in \mathbb{C}^{{R}_s}$ is the vector of channel gain of wireless link from RRHs in the $s^{th}$ slice to the $i^{th}$ UE in $v^{th}$ service. In addition, $\bold{w}_{R_s,u(v,i)} \in \mathbb{C}^{{R}_s}$ depicts the the transmit beamforming vector from RRHs in the $s^{th}$ slice to the $i^{th}$ UE in $v^{th}$ service. More over, $BN_0$ denotes the power of guassian additive noise and $I_{u(v,i)}$ is the power of interfering signals. 
To obtain SNR as formulated in equation \eqref{eq2}, let $\bold{y}_{U_v}$ be the received signal's vector of all users in $v^{th}$ service which is given by equation \eqref{eq3}
\begin{equation}\label{eq3}
\bold{y}_{U_v} = \sum_{k=1}^{N_k}\sum_{s = 1}^{N_s} \boldsymbol{H}^H_{\mathcal{R}_s,\mathcal{U}_v}(\boldsymbol{W}_{\mathcal{R}_s,\mathcal{U}_v}\boldsymbol{P}_{U_v}^{\frac{1}{2}}\boldsymbol{x}_{\mathcal{R}_s}+ \boldsymbol{q}_{R_s}) \zeta_{U_v,k}+ \boldsymbol{z}_{\mathcal{U}_v}
\end{equation}
where, $\boldsymbol{z}_{U_v}$ is the additive Gaussian noise $\boldsymbol{z_{U_v}} \backsim \mathcal{N}(0,N_0\boldsymbol{I}_{{U}_v})$ and $N_0$ is the noise power. Furthermore, $\zeta_{U_v,k} \in \{0,1\}$ is a binary parameter that map Physical Resource Blocks(PRB) to UE. 
  
\section*{References}

Please number citations consecutively within brackets \cite{b1}. The 
sentence punctuation follows the bracket \cite{b2}. Refer simply to the reference 
number, as in \cite{b3}---do not use ``Ref. \cite{b3}'' or ``reference \cite{b3}'' except at 
the beginning of a sentence: ``Reference \cite{b3} was the first $\ldots$''

Number footnotes separately in superscripts. Place the actual footnote at 
the bottom of the column in which it was cited. Do not put footnotes in the 
abstract or reference list. Use letters for table footnotes.

Unless there are six authors or more give all authors' names; do not use 
``et al.''. Papers that have not been published, even if they have been 
submitted for publication, should be cited as ``unpublished'' \cite{b4}. Papers 
that have been accepted for publication should be cited as ``in press'' \cite{b5}. 
Capitalize only the first word in a paper title, except for proper nouns and 
element symbols.

For papers published in translation journals, please give the English 
citation first, followed by the original foreign-language citation \cite{b6}.

\begin{thebibliography}{00}
\bibitem{b1} G. Eason, B. Noble, and I. N. Sneddon, ``On certain integrals of Lipschitz-Hankel type involving products of Bessel functions,'' Phil. Trans. Roy. Soc. London, vol. A247, pp. 529--551, April 1955.
\bibitem{b2} J. Clerk Maxwell, A Treatise on Electricity and Magnetism, 3rd ed., vol. 2. Oxford: Clarendon, 1892, pp.68--73.
\bibitem{b3} I. S. Jacobs and C. P. Bean, ``Fine particles, thin films and exchange anisotropy,'' in Magnetism, vol. III, G. T. Rado and H. Suhl, Eds. New York: Academic, 1963, pp. 271--350.
\bibitem{b4} K. Elissa, ``Title of paper if known,'' unpublished.
\bibitem{b5} R. Nicole, ``Title of paper with only first word capitalized,'' J. Name Stand. Abbrev., in press.
\bibitem{b6} Y. Yorozu, M. Hirano, K. Oka, and Y. Tagawa, ``Electron spectroscopy studies on magneto-optical media and plastic substrate interface,'' IEEE Transl. J. Magn. Japan, vol. 2, pp. 740--741, August 1987 [Digests 9th Annual Conf. Magnetics Japan, p. 301, 1982].
\bibitem{b7} M. Young, The Technical Writer's Handbook. Mill Valley, CA: University Science, 1989.
\end{thebibliography}
\vspace{12pt}
\color{red}
IEEE conference templates contain guidance text for composing and formatting conference papers. Please ensure that all template text is removed from your conference paper prior to submission to the conference. Failure to remove the template text from your paper may result in your paper not being published.

\end{document}
