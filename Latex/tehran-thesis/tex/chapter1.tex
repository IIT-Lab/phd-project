\chapter{مقدمه}
\section{مقدمه ای بر ساختار \lr{ORAN}}
\lr{Open RAN}(\lr{ORAN})
تبسیط و ترکیبی از دو ساختار \lr{C-RAN} و \lr{xRAN} می باشد که انتظار می رود که در فناوری نسل پنجم مخابرات مورد استفاده قرار گرفته و منجر به بهبود عملکرد شبکه های دسترسی رادیویی \lr{RAN} گردد. 
ایده اصلی \lr{C-RAN} جداسازی بخش رادیویی (\lr{RRH}) 
\LTRfootnote{Radio Remote Head}
 از واحد پردازشی باند پایه (\lr{BBU})
 \LTRfootnote{Baseband Unit}
  است.
از تجمیع \lr{BBU} ها بر روی سرور ابری، \lr{BBU-Pool} ایجاد می شود.
این ساختار جدید جزو یکی از ساختارهایی است که در \lr{5G} امکان استفاده را دارد.در این ساختار جدید در راستای بهینه سازی عملکرد \lr{BBU}
 ها در مواجهه باایستگاههای پایه پر ترافیک و کم ترافیک،
 \lr{BBU}ها به صورت یک مجموعه ی واحد تحت عنوان 
\lr{BBU Pool}
 در آمده اند که این مجموعه بین چندین سلول 
 به اشتراک گزارده شده و مطابق شکل زیر مجازی سازی
می شود. 
در توضیح بیشتر این ساختار می توان این گونه
عنوان کرد که \lr{BBU Pool} به عنوان یک خوشه ی مجازی
در نظر گرفته می شود که شامل پردازش گرهایی می باشد
که پردازش های باند پایه را انجام می دهند. ارتباط بین
  \lr{BBU}ها در ساختار های فعلی به شکل  $X_2$ برقرار می شود
که در این ساختار ارتباط بین خوشه ها از فرم جدید $X_2$
تحت عنوان  $X_2 +$برقرار می شود.
\newline
در شکل \ref{fig:C-RAN} ساختار کلی شبکه ی  \lr{C-RAN} در سیستم های
\lr{ LTE}
 نمایش داده شده است. همان طور که در شکل قابل
مشاهده می باشد ساختار کلی شبکه  \lr{C-RAN} به دو بخش
 \lr{backhaul} و \lr{fronthaul} تقسیم بندی شده است. بخش
 \lr{fronthaul}شبکه به مرحله ی اتصال سایت های \lr{ RRH}به
 به \lr{BBU Pool} به اتصال \lr{backhaul} و بخش \lr{BBU Pool}
هسته ی شبکه ی سیار اطلاق می شود. همان گونه که قبلا
ذکر شد  \lr{ RRH}ها در نزدیکی انتن نصب شده و از طریق
لینک های انتقالی نوری با پهنای باند وسیع و تاخیر کم به
پردازشگرهای قوی در  \lr{BBU}متصل می شوند. توسط این
لینک های انتقالی است که سیگنال های دیجیتالی باند
پایه از نوع \lr{IQ} بین \lr{RRH} و \lr{BBU} انتقال می یابند \cite{checko2015cloud}.
\begin{figure}[H]
  \centering
    \includegraphics[width=\textwidth]{./fig/CRAN}
  \caption{ساختار شبکه ی \lr{C-RAN} \cite{checko2015cloud}}
  \label{fig:C-RAN}
\end{figure}
\section{مزایای شبکه ی \lr{C-RAN}}

در این بخش قصد داریم مزایای شبکه ی \lr{C-RAN} و هدف از استفاده ی آن در \lr{5G} را بیان کنیم.
\newline
در هر دو نوع سلولهای ماکرو و میکرو، می توان از ساختار \lr{C-RAN} بهره برد. در حالت ماکرو، متمرکز کردن \lr{BBU} ها به صورت \lr{BBU Pool}، منجر به استفاده ی بهینه از \lr{BBU} ها و کاهش هزینه ی ایستگاه پایه \LTRfootnote{base station} می شود. همچنین منجر به کاهش مصرف توان و فراهم کردن انعطاف پذیری بیشتر در شبکه و تطبیق آن با ترافیک غیر یکسان می شود. علاوه بر این، باعث تبدیل سیگنال تداخل به سیگنال مفید تبدیل می شود. در ادامه این مزایا به صورت گسترده تر بیان می گردد.   
